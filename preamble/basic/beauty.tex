\usepackage{tcolorbox}
\tcbuselibrary{skins}
\tcbuselibrary{xparse}
\tcbuselibrary{theorems}

\usepackage{color}
\usepackage{xcolor}

\usepackage{titlesec}
\usepackage{etoolbox}

\definecolor{cyan}{rgb}{0, 1, 1}
\definecolor{magenta}{rgb}{1, 0, 1}
\definecolor{yellow}{rgb}{1, 1, 0}

\definecolor{TheoremColor}{RGB}{183, 28, 28}
\definecolor{TheoremBackColor}{RGB}{255, 235, 238}

\definecolor{DefinitionColor}{RGB}{13, 71, 161}
\definecolor{DefinitionBackColor}{RGB}{227, 242, 253}

\newcommand{\seclan}[1]{%
	\hspace{1em}\transparenttext{0.5}{#1}%
}

\newcommand{\transparenttext}[2]{%
  \begin{tikzpicture}[baseline=(X.base)]
    \node[inner sep=0pt, outer sep=0pt, text opacity=#1] (X) {#2};
  \end{tikzpicture}%
}

\newenvironment{question}[3][]{
	\begin{enumerate}[
		labelsep=2em,
		ref=\thequestion
	]
		\begin{tcolorbox}[
			bicolor,
			title=\textsf{#2},
			colframe=teal,
			colback=teal!12.5!white,
			colbacklower=white,
		]
			\refstepcounter{question}
			\item[\bfseries\thequestion.]{#1}
				{#3}
			\tcblower
}{
		\end{tcolorbox}
	\end{enumerate}
}

\newenvironment{mydefs}{
	\begin{description}
}{
	\end{description}
}

\newcounter{definition}[section]
\counterwithin{definition}{section}

\newenvironment{defbox}[2]{
	\stepcounter{definition}
	\begin{tcolorbox}[
		bicolor,
		title=\textbf{定义 \thedefinition #1}\hfill\textbf{#2},
		colframe=teal,
		colback=teal!12.5!white,
		colbacklower=white,
	]
}{
	\end{tcolorbox}
}

\counterwithin{theorem}{section}

\newtcbtheorem[use counter=theorem,number within=section]{newtheorembox}{定理}{
	enhanced,
	theorem name,
	separator sign={\hspace{0.5em}},
	arc=0pt,
	boxrule=0pt, toprule=0.5pt, bottomrule=0.5pt,
	left=4pt, right=4pt,
	titlerule=0pt,
	toptitle=0pt, bottomtitle=4pt, top=0pt,
	colframe=TheoremColor, colback=TheoremBackColor!25!white, coltitle=TheoremColor,
	title style={
		top color=TheoremBackColor,
		bottom color=TheoremBackColor!25!white,
		middle color=TheoremBackColor
	},
	fonttitle=\bfseries\sffamily\normalsize,
	before title={\hspace{-4pt}},
	fontupper=\normalsize,
	description font={\rmfamily},
	overlay unbroken and first={%
		\node[
			rounded corners=1pt,
			draw=TheoremColor,
			line width=0.4pt,
			fill=TheoremBackColor,
			anchor=north east
		] at ([xshift=-1pt]title.north west) {\bfseries\sffamily\textcolor{TheoremColor}{\thetcbcounter}};
	}
}{thm}

\newtcbtheorem[use counter=theorem,number within=section]{newdefbox}{定义}{
	enhanced,
	theorem name,
	separator sign={\hspace{0.5em}},
	arc=0pt,
	boxrule=0pt, toprule=0.5pt, bottomrule=0.5pt,
	left=4pt, right=4pt,
	titlerule=0pt,
	toptitle=0pt, bottomtitle=4pt, top=0pt,
	colframe=DefinitionColor, colback=DefinitionBackColor!25!white, coltitle=DefinitionColor,
	title style={
		top color=DefinitionBackColor,
		bottom color=DefinitionBackColor!25!white,
		middle color=DefinitionBackColor
	},
	fonttitle=\bfseries\sffamily\normalsize,
	before title={\hspace{-4pt}},
	fontupper=\normalsize,
	description font={\rmfamily},
	overlay unbroken and first={%
		\node[
			rounded corners=1pt,
			draw=DefinitionColor,
			line width=0.4pt,
			fill=DefinitionBackColor,
			anchor=north east
		] at ([xshift=-1pt]title.north west) {\bfseries\sffamily\textcolor{DefinitionColor}{\thetcbcounter}};
	}
}{def}

\newenvironment{theorembox}[2]{
	\stepcounter{theorem}
	\begin{tcolorbox}[
		bicolor,
		title=\textbf{定理 \thetheorem #1}\hfill\textbf{#2},
		colframe=orange,
		colback=orange!12.5!white,
		colbacklower=white,
	]
}{
	\end{tcolorbox}
}

\newcommand{\mydefboxillus}[4]{
	\tcbsidebyside[
		bicolor,
		title=\textbf{#1}\hfill\textbf{#2},
		colframe=teal,
		colback=teal!12.5!white,
		colbacklower=white,
		sidebyside adapt=right,
	]
	{#3}{#4}
}

\newlength\titleindent
\setlength\titleindent{3em}

\titleformat{\section}
  {\normalfont\Large\bfseries}{\llap{\parbox{\titleindent}{\thesection\hfill}}}{0em}{}
